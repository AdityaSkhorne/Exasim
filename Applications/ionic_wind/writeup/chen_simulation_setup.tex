\documentclass[12pt, a4paper]{report}
\usepackage[utf8]{inputenc}
\usepackage{graphicx}
\usepackage{caption}
\usepackage{palatino}
\usepackage{float}
\usepackage{subcaption}
% \usepackage{siunitx}
\usepackage{bm}
\setlength{\topmargin}{-0.5in}
\setlength{\topmargin}{0in}
\setlength{\headheight}{0in}
\setlength{\topsep}{0in}
\setlength{\textheight}{9in}
\setlength{\oddsidemargin}{0in}
\setlength{\evensidemargin}{0in}
\setlength{\textwidth}{6.5in}
\renewcommand{\thesection}{\arabic{section}}
\usepackage{gensymb}
\usepackage{amssymb}
\usepackage{hyperref}
% \usepackage{physics}
\usepackage{amsmath,mleftright}
% \usepackage{unicode-math} % also loads fontspec
\hypersetup{
    colorlinks=true,
    linkcolor=blue,
    filecolor=magenta,      
    urlcolor=cyan,
}

\usepackage{xcolor}
\definecolor{light-gray}{gray}{0.95}
\newcommand{\code}[1]{\colorbox{light-gray}{\texttt{#1}}}


\usepackage{bm}

\newcommand{\bv}{{\bm v}}
\newcommand{\bfo}{{\bm f}}
\newcommand{\bq}{{\bm q}}
\newcommand{\br}{{\bm r}}
\newcommand{\bPi}{{\bm \Pi}}
\newcommand{\bOmega}{{\bm \Omega}}
\newcommand{\be}{{\bm e}}

\newcommand{\ur}{{u_r}}
\newcommand{\ut}{{u_\theta}}
\newcommand{\uz}{{u_z}}



% For code block insertion
\usepackage{listings}
\usepackage{color}

\definecolor{dkgreen}{rgb}{0,0.6,0}
\definecolor{gray}{rgb}{0.5,0.5,0.5}
\definecolor{mauve}{rgb}{0.58,0,0.82}

\lstset{frame=tb,
  language=Matlab,
  aboveskip=3mm,
  belowskip=3mm,
  showstringspaces=false,
  columns=flexible,
  basicstyle={\small\ttfamily},
  numbers=none,
  numberstyle=\tiny\color{gray},
  keywordstyle=\color{blue},
  commentstyle=\color{dkgreen},
  stringstyle=\color{mauve},
  breaklines=true,
  breakatwhitespace=true,
  tabsize=3
}


\begin{document}

\title{Ionic Wind Simulation Notes}
\date{\today}
\maketitle

\noindent

\section{Objectives}
The objective of this research is to replicate the results of Chen et al. (2017), "A Self-Consistent Model of Ionic Wind Generation by Negative Corona Discharges in Air With Experimental Validation".

\section{Governing Equations}
The following equations constitute equations 1-5 in Chen et al. They will be solved simultaneously in Exasim using the Convection-Diffusion transport model (Model D). The newly-introduced DAE "subproblem" module will be used to separately solve the species transport/diffusion problem and the electrostatic problem.

\begin{equation}
    \frac{\partial n_e}{\partial t} + \nabla \cdot (-\mu_e\vec{E}n_e - D_e\nabla n_e) = \alpha n_e|\mu_e\vec{E}| -\eta n_e|\mu_e\vec{E}| - k_{ep}n_en_p
\end{equation}

\begin{equation}
    \frac{\partial n_p}{\partial t} + \nabla \cdot (\mu_p\vec{E}n_p - D_p\nabla n_p) = \alpha n_e|\mu_e\vec{E}| - k_{np}n_nn_p - k_{ep}n_en_p
\end{equation}

\begin{equation}
    \frac{\partial n_n}{\partial t} + \nabla \cdot (-\mu_n\vec{E}n_n - D_n\nabla n_n) = \eta n_e|\mu_e\vec{E}| - k_{np}n_nn_p
\end{equation}

\begin{equation}
    \nabla^2\Phi = -\frac{e(n_p-n_e-n_n)}{\epsilon}
\end{equation}

\begin{equation}
    \vec{E} = -\nabla\Phi
\end{equation}

Additionally, the one-way coupling equations from the EHD model to the gas dynamics model (N-S) is provided in Eqns. 6-8:

\begin{equation}
    f_{ehd} = e(n_p -n_e-n_n) \vec{E}
\end{equation}

\begin{equation}
    \nabla \cdot \vec{u} = 0
\end{equation}

\begin{equation}
    \rho_g \left( \frac{\partial \vec{u}}{\partial t} + \vec{u}\cdot\nabla\vec{u}  \right) = -\nabla p + \mu_v\nabla^2\vec{u} + f_{ehd}
\end{equation}

\section{Cylindrical Coordinates}

The gradient and divergence operators in cylindrical coordinates $(r,\theta,z)$ are given, for a scalar function $f$ and a vector field ${\bm f}$, by

\[
\begin{array}{rcl}
\nabla f  & = & \displaystyle \frac{\partial f}{\partial r} \be_r +  \frac{1}{r} \frac{\partial f}{\partial \theta} \be_{\theta} + \frac{\partial f}{\partial z} \be_z \\[2ex]
\nabla \cdot {\bm f}   & = & \displaystyle \frac{1}{r} \frac{\partial  (r f_r)}{\partial r}  +  \frac{1}{r}  \frac{\partial    f_\theta}{\partial \theta}  +  \frac{\partial f_z}{\partial z} \\[2ex]
\end{array}
\]

\noindent
If we assume axial symmetry all variable are function of $(r,z)$ only.

\section{ Fully Conservative Form}

Assuming axial symmetry, the above equations can be written as 

\begin{center}
\[
{\bm M} \frac{\partial {\bf U}}{\partial t} +  \frac{1}{r} \frac{\partial (r {\bf F}_r)}{\partial r} +  \frac{\partial {\bf F}_z}{\partial z} -  {\bf S}  = {\bf 0}
\]
\end{center}
Were
\begin{equation} 
{\bm U} = \left( \begin{array}{c} n_e \\ n_p \\ n_n \\ \Phi \end{array} \right),  \quad {\bm F}_r = \left( \begin{array}{c} - \mu_e n_e E_r - D_e (q_e)_r \\ \mu_p n_p E_r - D_p (q_p)_r \\ -\mu_n n_n E_r - D_n (q_n)_r \\ -E_r \end{array} \right),  \quad {\bm F}_z = \left( \begin{array}{c} - \mu_e n_e E_z - D_e (q_e)_z \\ -\mu_p n_p E_z - D_p (q_p)_z \\ -\mu_n n_n E_z - D_n (q_n)_z \\ -E_z \end{array} \right),
\end{equation}
\begin{equation} 
{\bm S} = \left( \begin{array}{c} \alpha n_e|\mu_e\vec{E}| -\eta n_e|\mu_e\vec{E}| - k_{ep}n_en_p \\ \alpha n_e|\mu_e\vec{E}| - k_{np}n_nn_p - k_{ep}n_en_p \\ \eta n_e|\mu_e\vec{E}| - k_{np}n_nn_p \\ -\frac{e(n_p-n_e-n_n)}{\epsilon} \end{array} \right),  \quad {\bm M} = \left( \begin{array}{cccc} 
1 & 0 & 0 & 0 \\
0 & 1 & 0 & 0 \\
0 & 0 & 1 & 0 \\
0 & 0 & 0 & 0 \end{array} \right)
\end{equation}
and
\begin{equation} 
{\bm Q} = \left( \begin{array}{cc} (q_e)_r & (q_e)_z \\ (q_p)_r & (q_p)_z \\ (q_n)_r & (q_n)_z \\ -E_r & -E_z \end{array} \right),  
\end{equation}
with 
\begin{equation} 
{\bm Q} = \nabla {\bm U}.
\end{equation}
Note that under the  axisymmetry assumption, the gradient operators in cylindrical and cartesian coordinates are the same.


\section{Weak form and the Divergence Theorem}
The elemental volume becomes $dV = r dr dz$ and for any ${\bf W}$ we can write the following weighted residual form



\begin{center}
\[\begin{array}{c} \displaystyle
\int_V \left({\bm M} \frac{\partial {\bf U}}{\partial t} +  \frac{1}{r} \frac{\partial (r {\bf F}_r)}{\partial r}  + \frac{\partial {\bf F}_z}{\partial z} +   {\bf S}\right) {\bf W} \ r dr dz  = 0 ,
\end{array} \]
\end{center}
or, integrating by parts, 
\begin{center}
\[\begin{array}{c}
\displaystyle \int_V \left(({\bm M} r) \frac{\partial {\bf U}}{\partial t}   +r{\bf S}\right) {\bf W} \ dr  dz  
\displaystyle + \int_S\left(  r {\bf F}_r {\bm n}_r +  r {\bf F}_z {\bm n}_z \right) {\bf W}  \, dS \\[4ex] - \displaystyle \int_V \left(  r {\bf F}_r\frac{\partial {\bf W}}{\partial r}  + r {\bf F}_z\frac{\partial {\bf W}}{\partial z} \right) \ dr  dz  = {\bf 0}
\end{array}
\]
\end{center}

Note that in the integrals in the second and third lines the {\em effective} $dV$ becomes $dr dz$. The only differences between cylindrical and cartesian coordinates thus

\begin{itemize}
\item Multiply $\bm M$ and $\bm S$ by $r$. Note that this will require a modified mass matrix.
\item Multiply ${\bm F}_r$ and ${\bm F}_z$ by $r$

\end{itemize}

\section{Nondimensionalization}

Because of the wide range of scales used in this problem, it is important to nondimensionalize the quantities being solved for to prevent numerical instability.

\subsection{Nondimensional groups}
The following nondimensional groups were chosen:

\begin{itemize}
    \item $n_e^* = \frac{n_e}{N_{max}}$
    \item $n_p^* = \frac{n_p}{N_{max}}$
    \item $n_n^* = \frac{n_n}{N_{max}}$
    \item $\vec{E^*} = \frac{\vec{E}}{E_{bd}}$
    \item $\Phi^* = \frac{\Phi}{E_{bd}r_{tip}}$
    \item $t^* = \frac{t\mu_eE_{bd}}{r_{tip}}$
    \item $r^* = \frac{r}{r_{tip}}$
    \item $z^* = \frac{z}{r_{tip}}$
\end{itemize}

\noindent
Where $r_{tip}$ is the needle tip radius of curvature, 220$\mu m$, and $E_{bd}$ is the breakdown electric field strength in air, $3\times10^6 \frac{V}{m}$.\\

\noindent
We can re-write the governing equations (section 2) using the nondimensional groups:

\begin{align*}
        \frac{\partial (n_e^* \mu_e E_{bd} N_{max})}{\partial (t^*r_{tip})} + \nabla \cdot (-\mu_e(\vec{E^*}E_{bd})(n_e^*N_{max}) - D_e\nabla (n_e^*N_{max})) = \\ \alpha (n_e^*N_{max})|\mu_e(\vec{E^*}E_{bd})| -\eta (n_e^*N_{max})|\mu_e(\vec{E^*}E_{bd})| - k_{ep}(n_e^*N_{max})(n_p^*N_{max})
\end{align*}

\begin{align*}
    \frac{\partial (n_p^* \mu_e E_{bd} N_{max})}{\partial (t^*r_{tip})} + \nabla \cdot (\mu_p(\vec{E^*}E_{bd})(n_p^*N_{max}) - D_p\nabla (n_p^*N_{max})) = \\ \alpha (n_e^*N_{max})|\mu_e(\vec{E^*}E_{bd})| - k_{np}(n_n^*N_{max})(n_p^*N_{max}) - k_{ep}(n_e^*N_{max})(n_p^*N_{max})
\end{align*}

\begin{align*}
    \frac{\partial (n_n^* \mu_e E_{bd} N_{max})}{\partial (t^*r_{tip})} + \nabla \cdot (-\mu_n(\vec{E^*}E_{bd})(n_n^*N_{max}) - D_n\nabla (n_n^*N_{max})) = \\ \eta (n_e^*N_{max})|\mu_e(\vec{E^*}E_{bd})| - k_{np}(n_n^*N_{max})(n_p^*N_{max})
\end{align*}

\begin{align*}
    \nabla^2(\Phi^*E_{bd}r_{tip}) = -\frac{eN_{max}(n_p^*-n_e^*-n_n^*)}{\epsilon}
\end{align*}


\noindent
For now, consider the homogeneous case with no source terms:

\begin{align*}
        \frac{\partial (n_e^* \mu_e E_{bd} N_{max})}{\partial (t^*r_{tip})} + \nabla \cdot (-\mu_e(\vec{E^*}E_{bd})(n_e^*N_{max}) - D_e\nabla (n_e^*N_{max})) = 0
\end{align*}



\begin{align*}
    \frac{\partial (n_p^* \mu_e E_{bd} N_{max})}{\partial (t^*r_{tip})} + \nabla \cdot (\mu_p(\vec{E^*}E_{bd})(n_p^*N_{max}) - D_p\nabla (n_p^*N_{max})) = 0
\end{align*}




\begin{align*}
    \frac{\partial (n_n^* \mu_e E_{bd} N_{max})}{\partial (t^*r_{tip})} + \nabla \cdot (-\mu_n(\vec{E^*}E_{bd})(n_n^*N_{max}) - D_n\nabla (n_n^*N_{max})) = 0
\end{align*}

\begin{align*}
    \nabla^2(\Phi^*E_{bd}r_{tip}) = 0
\end{align*}

\noindent
Simplifying:
\begin{align*}
        \frac{\partial n_e^* }{\partial t^*} + r_{tip} \nabla \cdot \left(-(\vec{E^*})(n_e^*) - \frac{{D_e}}{\mu_e E_{bd}}\nabla (n_e^*)\right) = 0
\end{align*}

\begin{align*}
    \frac{\partial n_p^*}{\partial t^*} + r_{tip}\nabla \cdot \left(\frac{\mu_p}{\mu_e}(\vec{E^*})(n_p^*) - \frac{D_p}{\mu_eE_{bd}}\nabla (n_p^*)\right) = 0
\end{align*}


\begin{align*}
    \frac{\partial n_n^*}{\partial t^*} + r_{tip}\nabla \cdot \left(-\frac{\mu_n}{\mu_e}(\vec{E^*})(n_n^*) - \frac{D_n}{\mu_e E_{bd}}\nabla (n_n^*)\right) = 0
\end{align*}


\begin{align*}
    E_{bd}r_{tip}\nabla^2\Phi^* = 0
\end{align*}

\noindent
We now treat the spatial derivatives:

\begin{align*}
    \nabla \cdot () = \frac{\partial()}{\partial r} + \frac{\partial()}{\partial z} = \frac{\partial()}{\partial (r^*r_{tip})} + \frac{\partial()}{\partial (z^*r_{tip})} = \frac{1}{r_{tip}}\left(\frac{\partial()}{\partial r^*} + \frac{\partial()}{\partial z^*}\right)
\end{align*}

\noindent
Resulting in:
\begin{align*}
        \frac{\partial n_e^* }{\partial t^*} +  \nabla \cdot \left(-(\vec{E^*})(n_e^*) - \frac{{D_e}}{r_{tip}\mu_e E_{bd}}\nabla (n_e^*)\right) = 0
\end{align*}

\begin{align*}
    \frac{\partial n_p^*}{\partial t^*} + \nabla \cdot \left(\frac{\mu_p}{\mu_e}(\vec{E^*})(n_p^*) - \frac{D_p}{r_{tip}\mu_eE_{bd}}\nabla (n_p^*N_{max})\right) = 0
\end{align*}

\begin{align*}
    \frac{\partial n_n^*}{\partial t^*} + \nabla \cdot \left(-\frac{\mu_n}{\mu_e}(\vec{E^*})(n_n^*) - \frac{D_n}{r_{tip}\mu_e E_{bd}}\nabla (n_n^*)\right) = 0
\end{align*}

\begin{align*}
    \nabla^2\Phi^* = 0
\end{align*}

\section{\bf \large Boundary Conditions}
Note: Boundary numbering follows the boundary numbering in the paper
\subsection{Boundary 1}

\begin{table}[!h]
    \centering\begin{tabular}{c|c|c}
        Equation & Boundary condition & Boundary condition type\\ \hline
        1 & Total flux $-\vec{n} \cdot\left(-\mu_{\mathrm{e}} \vec{E} -D_{\mathrm{e}} \nabla n_{\mathrm{e}}\right) =\gamma n_{\mathrm{p}}|\mu_{\mathrm{p}} \vec{E}|$ & Neumann \\ \hline
        2 & Outflow, $\vec{n} \cdot\left(-D_{\mathrm{p}} \nabla n_{\mathrm{p}}\right)=0$  & Neumann \\ \hline
        3 & $n_n=0$ & Dirichlet\\ \hline
        4 & $\Phi=-U_a$ & Dirichlet \\ \hline

    \end{tabular}
    \caption{Boundary conditions for the emitter tip (Boundary surface 1)}
\end{table}
\clearpage

\subsection{Boundary 2}
\begin{table}[!h]
    \centering\begin{tabular}{c|c|c}
        Equation & Boundary condition & Boundary condition type\\ \hline
        1 & Axial symmetry $\frac{\partial n_{\mathrm{e}}}{\partial r}=0$ &  Neumann\\ \hline
        2 & Axial symmetry $\frac{\partial n_{\mathrm{p}}}{\partial r}=0$  & Neumann \\ \hline
        3 & Axial symmetry $\frac{\partial n_{\mathrm{n}}}{\partial r}=0$ & Neumann\\ \hline
        4 & Axial symmetry $\frac{\partial \phi}{\partial r}=0 \quad$ & Neumann \\ \hline

    \end{tabular}
    \caption{Boundary conditions for  (Boundary surface 2)}
\end{table}

\subsection{Boundary 3}

\begin{table}[!h]
    \centering\begin{tabular}{c|c|c}
        Equation & Boundary condition & Boundary condition type\\ \hline
        1 & Open boundary $\begin{array}{c}\vec{n} \cdot\left(-D_{\mathrm{e}} \nabla n_{\mathrm{e}}\right)=0 ; \vec{n} \cdot\left(-\mu_{\mathrm{e}} \vec{E}\right) \geqslant 0 \\ n_{\mathrm{e}}=0 ; \vec{n} \cdot\left(-\mu_{\mathrm{e}} \vec{E}\right)<0\end{array} $ & Neumann/Dirichlet \\ \hline
        2 & Open boundary $\begin{array}{c}\vec{n} \cdot\left(-D_{\mathrm{p}} \nabla n_{\mathrm{p}}\right)=0 ; \vec{n} \cdot\left(-\mu_{\mathrm{p}} \vec{E}\right) \geqslant 0 \\ n_{\mathrm{p}}=0 ; \vec{n} \cdot\left(-\mu_{\mathrm{p}} \vec{E}\right)<0\end{array} $   & Neumann/Dirichlet \\ \hline
        3 & Open boundary $\begin{array}{c}\vec{n} \cdot\left(-D_{\mathrm{n}} \nabla n_{\mathrm{n}}\right)=0 ; \vec{n} \cdot\left(-\mu_{\mathrm{n}} \vec{E}\right) \geqslant 0 \\ n_{\mathrm{n}}=0 ; \vec{n} \cdot\left(-\mu_{\mathrm{n}} \vec{E}\right)<0\end{array} $  & Neumann/Dirichlet\\ \hline
        4 & Ground $\phi=0$ & Dirichlet \\ \hline

    \end{tabular}
    \caption{Boundary conditions for  (Boundary surface 3)}
\end{table}

\subsection{Boundary 4}

\begin{table}[!h]
    \centering\begin{tabular}{c|c|c}
        Equation & Boundary condition & Boundary condition type\\ \hline
        1 & Outflow $\vec{n} \cdot\left(-D_{\mathrm{e}} \nabla n_{\mathrm{e}}\right)=0\quad$ & Neumann \\ \hline
        2 & $n_p=0$  & Dirichlet \\ \hline
        3 & Outflow $\vec{n} \cdot\left(-D_{\mathrm{n}} \nabla n_{\mathrm{n}}\right)=0$ &Neumann \\ \hline
        4 & Ground $\phi=0$ & Dirichlet \\ \hline

    \end{tabular}
    \caption{Boundary conditions for  (Boundary surface 4)}
\end{table}
\clearpage

\subsection{Boundary 5 and 6}

\begin{table}[!h]
    \centering\begin{tabular}{c|c|c}
        Equation & Boundary condition & Boundary condition type\\ \hline
        1 & Open boundary $\begin{array}{c}\vec{n} \cdot\left(-D_{\mathrm{e}} \nabla n_{\mathrm{e}}\right)=0 ; \vec{n} \cdot\left(-\mu_{\mathrm{e}} \vec{E}\right) \geqslant 0 \\ n_{\mathrm{e}}=0 ; \vec{n} \cdot\left(-\mu_{\mathrm{e}} \vec{E}\right)<0\end{array} $ & Neumann/Dirichlet \\ \hline
        2 & Open boundary $\begin{array}{c}\vec{n} \cdot\left(-D_{\mathrm{p}} \nabla n_{\mathrm{p}}\right)=0 ; \vec{n} \cdot\left(-\mu_{\mathrm{p}} \vec{E}\right) \geqslant 0 \\ n_{\mathrm{p}}=0 ; \vec{n} \cdot\left(-\mu_{\mathrm{p}} \vec{E}\right)<0\end{array} $   & Neumann/Dirichlet \\ \hline
        3 & Open boundary $\begin{array}{c}\vec{n} \cdot\left(-D_{\mathrm{n}} \nabla n_{\mathrm{n}}\right)=0 ; \vec{n} \cdot\left(-\mu_{\mathrm{n}} \vec{E}\right) \geqslant 0 \\ n_{\mathrm{n}}=0 ; \vec{n} \cdot\left(-\mu_{\mathrm{n}} \vec{E}\right)<0\end{array} $  & Neumann/Dirichlet\\ \hline
        4 & Zero charge $\vec{n} \cdot\left(\epsilon \vec{E}\right)<0$ & Neumann \\ \hline

    \end{tabular}
    \caption{Boundary conditions for  (Boundary surfaces 5 and 6)}
\end{table}

\section{\bf \large Code Review}


\large \texttt{pdeapp.m}
\begin{lstlisting}
    % clear exasim data from memory
    clear pde mesh master dmd sol;

    % Add Exasim to Matlab search path
    cdir = pwd(); ii = strfind(cdir, "Exasim");
    run(cdir(1:(ii+5)) + "/Installation/setpath.m");

    % create pde and mesh for each PDE model
    pdeapp1;
    pdeapp2;

    % call exasim to generate and run C++ code to solve the PDE models
    [sol,pde,mesh,master,dmd,compilerstr,runstr] = exasim(pde,mesh);

    % visualize the numerical solution of the PDE model using Paraview
    for m = 1:length(pde)
        pde{m}.visscalars = {"temperature", 1};  % list of scalar fields for visualization
        pde{m}.visvectors = {"temperature gradient", [2 3]}; % list of vector fields for visualization
        pde{m}.visfilename = "dataout" + num2str(m) + "/output";  
        vis(sol{m},pde{m},mesh{m}); % visualize the numerical solution
    end
\end{lstlisting}

\clearpage

\large \texttt{pdeapp1.m}
\begin{lstlisting}
    % Physical parameters
    Kep = 2e-13;             % mu[1] Recombination coeff - pos and neg ions [m^3/s]
    Knp = 2e-13;             % mu[2] Recombination coeff - pos ions and electrons [m^3/s]
    mu_p = 2.43e-4;          % mu[3] Pos ion mobility [m^2/(Vs)]
    mu_n = 2.7e-4;           % mu[4] Neg mobility [m^2/(Vs)]
    De = 0.18;               % mu[5] Electron diffusion coefficient [m^2/s]
    Dp = 0.028e-4;           % mu[6] Pos ion diffusion coefficient [m^2/s]
    Dn = 0.043e-4;           % mu[7] Neg diffusion coefficient [m^2/s]
    Nmax = 1e16;             % mu[8] Max number density for initial charge distribution [particles/m^3]
    r0 = 0.0;                % mu[9] r-pos of emitter tip in reference frame [m]
    z0 = 0.045;              % mu[10]z-pos of emitter tip in reference frame [m]
    s0 = 1e-2;               % mu[11]Std deviation of initial charge distribution [m]
    e = 1.6022e-19;          % mu[12]Charge on electron [C]
    epsilon = 8.854e-12;     % mu[13]absolute permittivity of air [C^2/(N*m^2)]
    Ua = -10e3;              % mu[14]Emitter potential relative to ground [V]
    gamma = 0.001;           % mu[15]Secondary electron emission coefficient [1/m]
    E_bd = 3e6;              % mu[16]Breakdown E field in air [V/m]
    r_tip = 220e-6;          % mu[17] Tip radius of curvature [m]

    % Set discretization parameters, physical parameters, and solver parameters
                        %    1   2    3     4     5  6   7    8    9   10  11  12   13     14   15     16     17
    pde{1}.physicsparam = [Kep, Knp, mu_p, mu_n, De, Dp, Dn, Nmax, r0, z0, s0, e, epsilon, Ua, gamma, E_bd, r_tip];



    % Mesh
    [mesh{1}.p,mesh{1}.t] = gmshcall(pde{1}, "chen_geom_coarse.msh", 2, 0);

    % expressions for domain boundaries
    eps = 1e-4;
    xmin = min(mesh{1}.p(1,:));
    xmax = max(mesh{1}.p(1,:));
    ymin = min(mesh{1}.p(2,:));
    ymax = max(mesh{1}.p(2,:));
    x2 = 0.017;
    x3 = 0.015;
    
    bdry1 = @(p) (p(1,:) < xmin+eps);    % axis symmetric boundary            
    bdry2 = @(p) (p(1,:) > xmax - eps);  % open boundary 1                                    
    bdry3 = @(p) (p(2,:) > ymax - eps);  % open boundary 2                                     
    bdry4 = @(p) (p(2,:) < ymin+eps) && (p(1,:) < x3+eps);   % grounded boundary - open                                       
    bdry5 = @(p) (p(2,:) < ymin+eps) && (p(1,:) > x2-eps);   % grounded boundary                                    
    bdry6 = @(p) (p(2,:) < 0.04);                            % grounded boundary - cylinder
    bdry7 = @(p) (p(1,:) < x2+eps);                          % needle tip          
    
    mesh{1}.boundaryexpr    = {bdry1, bdry2, bdry3, bdry4, bdry5, bdry6, bdry7};
    mesh{1}.boundarycondition = [2,     5,     5,     3,     4,     4,    1]; % Set boundary condition for each boundary



    % Solver configuration parameters
    pde{1}.porder = 2;          % polynomial degree
    pde{1}.tau = 1.0;           % DG stabilization parameter
    pde{1}.NLtol = 1.0e-6;
    pde{1}.linearsolvertol = 1.0e-4;
    pde{1}.ppdegree = 20;
    pde{1}.precMatrixType = 2;
    
    % solver parameters
    pde{1}.torder = 1;          % time-stepping order of accuracy
    pde{1}.nstage = 1;          % time-stepping number of stages
    pde{1}.dt = 1.0e-7*ones(1,3);   % time step sizes
    pde{1}.visdt = 1.0e-6;        % visualization timestep size
    pde{1}.soltime = 1:pde{1}.visdt:length(pde{1}.dt); % steps at which solution are collected
    pde{1}.GMRESrestart=25;            % number of GMRES restarts
    pde{1}.linearsolveriter=50;        % number of GMRES iterations
    pde{1}.NLiter=2;                   % Newton iterations
    
    % set indices to obtain v from the solutions of the other PDE models 
    % first column : model index
    % second column: solution index
    pde{1}.vindx = [2 1; 2 2; 2 3];     % check this
    pde{1}.subproblem = 1;
    
\end{lstlisting}

\clearpage
\large \texttt{pdeapp2.m}
\begin{lstlisting}
    % set indices to obtain v from the solutions of the other PDE models 
    % first column : model index
    % second column: solution index
    pde{2}.porder = 2;          % polynomial degree
    pde{2}.vindx = [1 1];
    pde{2}.subproblem = 1;

    pde{2}.NLtol = 1.0e-6;
    pde{2}.linearsolvertol = 1.0e-4;
    pde{2}.ppdegree = 20;
    pde{2}.precMatrixType = 2;

    % solver parameters
    pde{2}.torder = 1;          % time-stepping order of accuracy
    pde{2}.nstage = 1;          % time-stepping number of stages
    pde{2}.dt = 1.0e-7*ones(1,3);   % time step sizes
    pde{2}.visdt = 1.0;        % visualization timestep size
    pde{2}.soltime = 1:pde{2}.visdt:length(pde{2}.dt); % steps at which solution are collected
    pde{2}.GMRESrestart=25;            % number of GMRES restarts
    pde{2}.linearsolveriter=50;        % number of GMRES iterations
    pde{2}.NLiter=2;                   % Newton iterations
\end{lstlisting}

\vspace{5mm}
\texttt{pdemodel1.m}
\begin{lstlisting}
    function pde = pdemodel1
    pde.mass = @mass;
    pde.flux = @flux;
    pde.source = @source;
    pde.fbou = @fbou;
    pde.ubou = @ubou;
    pde.initu = @initu;
    pde.initw = @initw;
    end
    
    function m = mass(u, q, w, v, x, t, mu, eta)
        r = x(1);
        m = r*[1.0, 1.0, 1.0];        % Multiply by r for axisymmetric
    end
    
    function s = source(u, q, w, v, x, t, mu, eta)
        s = sym([0, 0, 0, 0]);
    end
    
    function f = flux(u, q, w, v, x, t, mu, eta)
    
        r = x(1);
    
        disp(size(w))
        Ex = w(2);       % Check to make sure the sign is correct
        Ey = w(3);
    
        mu_e = 1.9163*((Ex^2 + Ey^2)^0.5)^(-0.25);     % Ionization coefficient [1/m]
        mu_p = mu(3);
        mu_n = mu(4);
        De = mu(5);
        Dp = mu(6);
        Dn = mu(7);
        E_bd = mu(16);
        r_tip = mu(17);
    
        ne = u(1);
        np = u(2);
        nn = u(3);
    
        % Nondimensional groups
        A1 = 1;
        B1 = De/(r_tip*mu_e*E_bd);
        
        A2 = mu_p/mu_e;
        B2 = Dp/(r_tip*mu_e*E_bd);
        
        A3 = mu_n/mu_e;
        B3 = Dn/(r_tip*mu_e*E_bd);
        
        %%%% Eqn 1
        %%% Note + sign in '+De*q' because q = -grad(u)
        f11 = -A1*ne*Ex +B1*q(1);    % x
        f12 = -A1*ne*Ey +B1*q(4);    % y
    
        %%%% Eqn 2
        f21 = A2*np*Ex +B2*q(2);     % x
        f22 = A2*np*Ey +B2*q(5);     % y
    
        %%%% Eqn 3
        f31 = -A3*nn*Ex +B3*q(3);     % x
        f32 = -A3*nn*Ey +B3*q(6);     % y
        
        fx = [f11 f21 f31];
        fy = [f12 f22 f32];
        f = r*[fx(:) fy(:)];
    end
    
    function fb = fbou(u, q, w, v, x, t, mu, eta, uhat, n, tau)
        % pde fluxes
        f = flux(u, q, w, v, x, t, mu, eta);    
        
        % numerical flux
        fh = f(:,1)*n(1) + f(:,2)*n(2) + tau*(u - uhat);
    
        ne = u(1);
        np = u(2);
        nn = u(3);
        disp(size(w))
        Ex = w(2);       % Check the dimension of w and make sure the sign is correct
        Ey = w(3);
        
        mu_e = 1.9163*((Ex^2 + Ey^2)^0.5)^(-0.25);     % Ionization coefficient [1/m]
        mu_p = mu(3);
        mu_n = mu(4);
        gamma = mu(15);
        
        % inviscid fluxes
        fix = [-ne*Ex (mu_p/mu_e)*np*Ex -(mu_n/mu_e)*nn*Ex]; fix = fix(:);     % Nondimensionalized
        fiy = [-ne*Ey (mu_p/mu_e)*np*Ey -(mu_n/mu_e)*nn*Ey]; fiy = fiy(:);
        fih = fix*n(1) + fiy*n(2); %+ tau*(u - uhat); check
        
        % boundary flux on the needle tip - Chen boundary 1
        fb1 = fh;
        fb1(1) = -gamma*np*(((mu_p/mu_e)*Ex)^2 + ((mu_p/mu_e)*Ey)^2)^0.5;            % Nondimensionalized   check sign here (-n)
        fb1(2) = fih(2);
    
        % axis symmetric boundary condition   - axisymmetric BC means the diffusive flux=0
        fb2 = [0; 0; 0]; 
        
        % Chen bdry 3
        En = Ex*n(1) + Ey*n(2);
        signEn = tanh(1e3*(-En));    
        alpha = 0.5 + 0.5*signEn;   % Alpha=1 when (-En) is positive: 
        fb3 = alpha*fih + (1-alpha)*fh;        % Check for this sign being flipped in the switch
        
        % Grounded boundary -> 4 in paper
        fb4 = fih;
        fb4(2) = fh(2);
            
        fb = [fb1 fb2 fb3 fb4 fb3];     % Note: For eqns 1-3, BCs 3,5&6 are the same (open boundary)
    end
    
    function ub = ubou(u, q, w, v, x, t, mu, eta, uhat, n, tau)
        E_bd = mu(16);
        r_tip = mu(17);
        Ua = mu(14);
    
        % needle tip - boundary 1
        ub1 = u;
        ub1(3) = 0;
        
        % axis symmetric boundary 2
        ub2 = u;
        
        % grounded boundary
        Ex = w(2);       % Check to make sure the sign is correct
        Ey = w(3);
        En = Ex*n(1) + Ey*n(2);
        signEn = tanh(1e3*(-En));    
        alpha = 0.5 + 0.5*signEn;
        ub3 = alpha*u + (1-alpha)*[0,0,0];
       
        ub4 = u;
        ub4(2) = 0;    
            
        ub = [ub1 ub2 ub3 ub4 ub3];     % Note: For eqns 1-3, BCs 3,5&6 are the same (open boundary)
    end
    
    function u0 = initu(x, mu, eta)
        r = x(1);
        z = x(2);
    
        r0 = mu(9);
        z0 = mu(10);
        s0 = mu(11);
    
        g = exp(-(r-r0)^2/(2*s0^2) - (z-z0)^2/(2*s0^2));       % Nondimensionalized by N_max
    
        % Eqn 1
        u1_0 = g;
    
        % Eqn 2
        u2_0 = g;
    
        % Eqn 3
        u3_0 = 0;
    
        u0 = [u1_0, u2_0, u3_0];
    end
    
    function w0 = initw(x, mu, eta)
        w0 = sym([0; 0; 0]);
    end    
\end{lstlisting}

\vspace{5mm}
\texttt{pdemodel2.m}
\begin{lstlisting}
    function pde = pdemodel2
    pde.mass = @mass;
    pde.flux = @flux;
    pde.source = @source;
    pde.fbou = @fbou;
    pde.ubou = @ubou;
    pde.initu = @initu;
    pde.initw = @initw;
    end
    
    function m = mass(u, q, w, v, x, t, mu, eta)
        m = sym(0.0);        % Multiply by r for axisymmetric
    end
    
    function f = flux(u, q, w, v, x, t, mu, eta)
    f = [-q(1) -q(2)];      % Check the sign here!
    end
    
    function s = source(u, q, w, v, x, t, mu, eta)
        s = sym(0.0);
    end
    
    function fb = fbou(u, q, w, v, x, t, mu, eta, uhat, n, tau)
        f = flux(u, q, w, v, x, t, mu, eta);
        fh = f(1)*n(1) + f(2)*n(2) + tau*(u(1)-0.0);
    
        fb = [fh 0 fh fh 0];
    end
    
    function ub = ubou(u, q, w, v, x, t, mu, eta, uhat, n, tau)
        E_bd = mu(16);
        r_tip = mu(17);
        Ua = mu(14);
        ub = [-Ua/(E_bd*r_tip), w(1), 0, 0, w(1)];    % Need to check that the solution variable for the current equation is w here and not u
    end
    
    function u0 = initu(x, mu, eta)
        u0 = sym(0.0);
    end
    
    function w0 = initw(x, mu, eta)
        w0 = sym(0.0);
    end    
\end{lstlisting}



\begin{figure}[!h]
    \centering
    \includegraphics*[width=.8\linewidth]{indicator_fcn.png}
    \caption{Indicator function for $E \cdot n$ used in this problem}
\end{figure}



\begin{figure}[!h]
    \centering
    \includegraphics*[width=.99\linewidth]{exasim_bc_index.png}
    \caption{Computational domain of Chen 2017 with Exasim BC indices overlaid}
    \label{<label>}
\end{figure}

\end{document}